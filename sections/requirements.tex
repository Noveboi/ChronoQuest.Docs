\section{Ανάλυση Απαιτήσεων}
Λειτουργικές απαιτήσεις ορίζουμε χαρακτηριστικά του συστήματος. Για ευκολία στην αναφορά μιας απαίτησης, κωδικοποιούμε την καθεμιά \textlatin{FR-XX-NNN}, όπου το \textbf{ΧΧ} είναι ένα πλαίσιο του πεδίου στο οποίο η απαίτηση αναφέρεται και το \textbf{NNN} είναι ο αριθμός της απαίτησης σχετικά με το πλαίσιο \textbf{ΧΧ}.

\subsection{Χρήστες}
\begin{itemize}
    \item \textbf{\textlatin{FR-USR-001}}: Ένας χρήστης θα μπορεί να συνδεθεί στο σύστημα με την χρήση του \textlatin{e-mail} και του κωδικού πρόσβασης του
    \item \textbf{\textlatin{FR-USR-002}}: Ένας χρήστης θα μπορεί να διαβάσει "θεωρία" για κάθε κεφάλαιο
    \item \textbf{\textlatin{FR-USR-003}}: Ένας χρήστης θα έχει τη δυνατότητα να απαντήσει στα κουίζ χωρίς περιορισμό (δεν απαιτείται να έχει πρώτα διαβάσει τη θεωρία)
    \item \textbf{\textlatin{FR-USR-004}}: Ένας χρήστης θα μπορεί να δει τα στατιστικά του
    \item \textbf{\textlatin{FR-USR-005}}: Ένας χρήστης θα μπορεί να γράφει διαγώνισμα όταν έχει ολοκληρώσει όλα τα κουίζ
\end{itemize}

\subsection{Ενότητες}
\begin{itemize}
    \item \textbf{\textlatin{FR-CH-001}}: Μια ενότητα θα περιέχει θεωρία και ένα κουίζ αξιολόγησης της
    \item \textbf{\textlatin{FR-CH-002}}: Μια ενότητα θα θεωρείται ολοκληρωμένη, αν έχει ολοκληρωθεί το κουίζ της
    \item \textbf{\textlatin{FR-CH-003}}: Θα μπορεί ο χρήστης να προσπεράσει ενότητες (ελεύθερη πλοήγηση)
    \item \textbf{\textlatin{FR-CH-004}}: Μια ενότητα θα σχετίζεται με ένα και μόνο ένα θέμα.
\end{itemize}

\subsection{Κουίζ}
\begin{itemize}
    \item \textbf{\textlatin{FR-QZ-001}}: Το κουίζ θα περιέχει ερωτήσεις αξιολόγησης στην εκάστοτε ενότητα που βρίσκεται
    \item \textbf{\textlatin{FR-QZ-002}}: Το κουίζ θα έχει σταθερή δυσκολία (δεν θα υπάρχουν επίπεδα δυσκολίας)
    \item \textbf{\textlatin{FR-QZ-003}}: Το κουίζ δεν θα έχει χρονικό περιορισμό
    \item \textbf{\textlatin{FR-QZ-004}}: Μπορεί ο χρήστης να πλοηγηθεί ελεύθερα μεταξύ των ερωτήσεων χωρίς να έχει δώσει απάντηση
    \item \textbf{\textlatin{FR-QZ-005}}: Το κουίζ θα δίνει τη δυνατότητα στους χρήστες με πολύ  καλή απόδοση στις ερωτήσεις να παραλείψουν τις ερωτήσεις που είναι "\textlatin{skippable}"
    \item \textbf{\textlatin{FR-QZ-006}}: Όταν ο χρήστης κάνει ένα κουίζ, δεν θα μπορεί να πάει σε οποιαδήποτε ενότητα μέχρις ότου το ολοκληρώσει
\end{itemize}

\subsection{Ερωτήσεις}
\begin{itemize}
    \item \textbf{\textlatin{FR-QUEST-001}}: Οι ερωτήσεις θα είναι πολλαπλής επιλογής
    \item \textbf{\textlatin{FR-QUEST-002}}: Οι ερωτήσεις θα έχουν συγκεκριμένο αριθμό επιλογών (τέσσερις - 4)
    \item \textbf{\textlatin{FR-QUEST-003}}: Θα πρέπει να φανερώνεται η σωστή απάντηση στην ερώτηση εάν απαντηθεί λάθος.
    \item \textbf{\textlatin{FR-QUEST-004}}: Μια ερώτηση θα σχετίζεται με ένα και μόνο ένα θέμα
\end{itemize}

\subsection{Ενισχυτικό Υλικό}
\begin{itemize}
    \item \textbf{\textlatin{FR-EXTRA-001}}: Το ενισχυτικό υλικό θα είναι προσαρμοσμένο για κάθε χρήστη
    \item \textbf{\textlatin{FR-EXTRA-002}}: Το ενισχυτικό υλικό θα μπορεί να καλύπτει όλα τα θέματα (Ιστορία, Γεωγραφία, Πολιτισμός)
    \item \textbf{\textlatin{FR-EXTRA-003}}: Το ενισχυτικό υλικό θα περιέχει κείμενο μόνο για θέματα που ο χρήστης έχει χαμηλή απόδοση
\end{itemize}

\subsection{Διαγώνισμα}
\begin{itemize}
    \item \textbf{\textlatin{FR-EXAM-001}}: Το διαγώνισμα θα έχει περιορισμένη χρονική διάρκεια
    \item \textbf{\textlatin{FR-EXAM-002}}: Το διαγώνισμα θα έχει μεταβλητή δυσκολία με βάση την απόδοση του χρήστη
    \item \textbf{\textlatin{FR-EXAM-003}}: Το διαγώνισμα θα έχει ερωτήσεις διαφορετικής δυσκολίας ανάλογα τον εκάστοτε χρήστη
    \item \textbf{\textlatin{FR-EXAM-004}}: Το διαγώνισμα θα έχει ερωτήσεις μεταβλητής δυσκολίας (εύκολη, μέτρια, δύσκολη)
    \item \textbf{\textlatin{FR-EXAM-005}}: Όταν ο χρήστης γράφει ένα διαγώνισμα, δεν θα μπορεί να πάει σε οποιοδήποτε ενότητα μέχρις ότου το ολοκληρώσει
    \item \textbf{\textlatin{FR-EXAM-006}}: Το διαγώνισμα θα δίνεται στο χρήστη αν μόνο και μόνο αν έχει ολοκληρώσει όλες τις ενότητες (διάβασμα και κουίζ) και το ενισχυτικό υλικό (αν του δόθηκε)
    \item \textbf{\textlatin{FR-EXAM-007}}: Το διαγώνισμα θα έχει μεταβλητή χρονική διάρκεια ανάλογα με την απόδοση του χρήστη στις ενότητες
\end{itemize}